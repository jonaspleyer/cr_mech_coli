General comments on \acp{abm} and their problems
\begin{itemize}
    \item \aclp{abm} are used for many purposes
    \item describe current way of estimating parameters; compare distributions to each other; no idea if
          individual parameters are correct
    \item challenge: estimate their parameters
    \item challenge: construct minimal models (requires flexibility)
\end{itemize}

General remarks about the steps in this work
\begin{itemize}
    \item Use rod-shaped bacteria as model system
    \item Clear separation between different components (link to overview)
    \item This separation allows us to generalize the type of approach and apply to other models
    \item think of as a "blueprint" for estimating parameters for \acp{abm}
\end{itemize}

Experimental observations: rod-shaped bacteria
\begin{itemize}
    \item growth + division "distinct mechanical properties", MreB
    \item bendiness (not much studied)
    \item individual parameters (i.e. growth rates)
    \item adhesion to each other
\end{itemize}

Computational studies of rod-shaped bacteria
\begin{itemize}
    \item not so many models for rod-shaped bacteria
    \item could include more effects such as bending
    \item almost no efforts to estimate parameters with individual-based models
    \item although many things known no "generalized mechanical model" constructed for rod-shaped bacteria
\end{itemize}

\section{Introduction (old)}

The last decades have fundamentally challenged and changed our understanding of how bacteria retain
their shape.
They exist in many physical forms such as spheroidal, rod-shaped or
spiral~\cite{Zapun2008,Young2006}.
Rod-shaped bacteria can grow by extending their cylindrical part, or by inserting new material at
the tip of the rod.
Species such as \textit{E.coli} and \textit{B.subtilis}~\cite{Errington2020} fall under the first
category while \textit{S.pombe} represents the latter.
New material is inserted in small bursts on the nanoscale in forms of patches, bands or
hoops~\cite{DePedro2003}.
Despite their differences such as wall-size or length \textit{B.subtilis} and \textit{E.coli} follow
growth rules which are comparable with respect to the extension of the rod~\cite{Chang2014}.

\paragraph{Experimental studies of mechanical properties}
It was long believed that bacteria lack cytoskeletal filaments but with the end of the last
century, it was discovered that the MreB protein fills this role since it "[..] forms an actin-like
cytoskeleton in bacteria [..]" \cite{Erickson2001}.
It can polymerize and form filaments which behave similar to actin microfilaments \cite{Dersch2020}.
Studies of its crystal structure further showed the similarity to actin \cite{vandenEnt2001} and the
MreB, MreC and MreD proteins were identified as homologues of actin \cite{Lowe2017_lj}.\
Jones et al. studied the species of different bacterial sub-kingdoms which have the MreB
gene~\cite{Jones2001}.
They concluded: "Many of the organisms are rod shaped, similar to \textit{B. subtilis} and
\textit{E. coli}.
Organisms with more complex shapes, including curved, filamentous, and helical bacteria, were all
abundantly represented." \cite{Jones2001}.
Furthermore, \cite{Wachi1987} found that mutants of \textit{E.Coli} which create defective MreB
proteins will be spherical instead of their natural rod-shaped form.

% \begin{itemize}
%     \item \cite{Robert2014} growth of individual rods, distribution of parameters
%     \item \cite{Koutsoumanis2013} inheritance of growth rates, growth parameter distributions
%     \item \cite{Ursell2014} MreB acts againts curvature during growth of cell
%     \item \cite{Takeuchi2005} cells in microchambers leads to plastic deformations
%     \item \cite{Amir2014} bending stiffness, plastic + elastic deformations
% \end{itemize}
Other experiments found that individual rods of \textit{E.Coli} grow exponentially until the
division event which is triggered by a size-sensing rather than a time-sensing
mechanism~\cite{Robert2014}.
These results also showed that the growth parameters were randomly distributed but did not quantify
said distribution.
Using $220$ \textit{Salmonella enterica}, Koutsoumanis et al. analysed the distribution of division
times and was able to quantify their distributions~\cite{Koutsoumanis2013}.
They found that while the initial division times were longer, there was no correlation between
division times of mother and daughter cells.
The first finding can be explained by the bacterias lag-phase~\cite{Bertrand2019} which is a
temporary period during which bacteria do not grow.
The second observation is an example of principle of thought which underlyies guides this work:
Cells are individual agents with individual parameters and thus individual behaviour and need to be
treated as such.
This is especially important if we want to determine parameters of our model for individual agents.

% \paragraph{Collective Effects \& Self-Organization}
% are emergent phenomena that result from the combined interactions of multiple cells.
% Although they do not allow us to fundamentally understand, the individual behaviour of cells
% directly, they can provide a way to falsify our results and thus further guide our modeling
% approach.
% \cite{Trejo2013} investigated the mechanical response of a tightly packed colony of \textit{Bacillus
% Subtilis}.
Trejo et al. studied the formation of macroscopic wrinkles when growing bacteria in a confined
space~\cite{Trejo2013}.
They were able to model the strain and stress of the collection of bacteria which requires that
cells are connected by a (to first order) spring-like force.
The developing wrinkles could then be described by a buckling instability.

Duvernoy et al. used laser techniques and force microscopy to investigate adhesion of individual
\textit{E.coli} and \textit{Pseudomonas aeruginosa} cells~\cite{Duvernoy2018}.
They found that the adhesion forces of the rods are polar, resulting in mechanical tension which in
turn determines daughter cell arrangement.
% They measured the size of the grown bacterial colonies, fitted ellipses to their shapes and found
% that adhesion and polar adhesion in particular resulted in a greater difference between the long and
% short axis of the fitted ellipse and thus a more oval shape rather than circular (WT).
By extending the work of \cite{Grant2014}, they further showed, that the critical value of
transitioning from a monolayer to multiple layers of bacteria, depends on the adhesive strength and
polarity of the interaction.
Grant et al. described the growing bacterial colony as a disc with radius $R$ which is pressed
into the agarose slab~\cite{Grant2014}.
The authors created a purposely-designed simulation in `C++` in order to model the effects of the
gel on the collection of bacteria.

Van Gestel et al. explored how \textit{B.subtilis} migrates over a surface by forming multicellular
structures~\cite{vanGestel2015}.
The cells organize themselves into "van Gogh bundles" of cells which are tightly aligned in chains
and form filamentous loops.
This phenomenon occurs at the border of the bacterial colony and the migration is driven by two
phenotypically different cell types~\cite{Lpez2010}.
% \begin{itemize}
%     \item \cite{Trejo2013} mechanical response of bacterial colony
%     \item \cite{Duvernoy2018} bacterial colonies form multilayers, interactions are polar
%     \item \cite{vanGestel2015} van Gogh bundles form from interplay of 2 cell-types
% \end{itemize}

\paragraph{Computational Modeling Frameworks which support Rod-Shaped Bacteria}
% \textbf{Requirements}
% \begin{itemize}
%     \item capture bendiness; do not assume rigid bodies
%     \item model in 2D and 3D possible
%     \item individual-based model; model single cells
%     \item collective phenomena; describe many interacting cells, not just a single one
% \end{itemize}
There have been many variations in modeling rod-shaped bacteria.
In order to provide enough flexibility and generality, we are in particular interested in models
that can capture non-rigid rods and work both in 2D and 3D scenarios.
Furthermore, we require that our model can describe bacteria individually and that a collective
model is simply the propduct of multiple individual agents interacting with each other.
In order to be able to investigate more specialized models, we need to be able to adjust and modify
any existing model.

In general, there are two classes of existing models which could be considered to fit these
criteria: purpose-built (onf-off) solutions and computational modeling frameworks which have been
designed to provide predefined cellular representations for various use-cases~\cite{Pleyer2023}.
However, explicit support for rod-shaped bacteria is rather sparse.
Biocellion has support for cylindrically-shaped agents but does not provide any mechanisms of how to
model bendiness or flexibility of the rod~\cite{Kang2014}.
BSim2.0 represents cells as rigid capsular cells made from a cylindrical center part and two
half-spheres, which are placed at the ends of the cylinder to round out the
shape~\cite{Matyjaszkiewicz2017}.
In order to calculate interactions between cells, possible overlaps are determined and minimized,
thus determining the position values of the next iteration step.
However BSim does not consider bending of the rods.
The \texttt{gro} programming language was designed to simulate the growth of colonies and cell-cell
communication~\cite{Gutirrez2017}.
Its “physics computation has been optimized for rigid rod-shaped bodies, like E. coli
bacteria”~\cite{Gutirrez2017}.
They recognize two types of forces which are acting on the cellular agents:
Local forces which are calculated between adjacent bacteria and a global force which pushes bacteria
outwards of the colony.
The latter of these is a phenomenological implementation of the observed colony expansion and the
associated central pressure with it.
This assumption may yield incorrect results for sparsely populated cases.
The engine is limited to 2D and does not consider polar interactions or bending of the rods.

% \textbf{Computational Modeling Frameworks}
% What has been done so far?
% \begin{itemize}
%     \item In general: not many frameworks which support some form of rod-shaped bacteria
%     \item \cite{Kang2014} Biocellion has support for cylindrically-shaped agents; no
%         bendiness/rigidity or polar interactions
%     \item \cite{Matyjaszkiewicz2017} BSim2.0 represents cells as rigid capsular cells made from a
%         cylindrical center part and two half-spheres, which are placed at the ends of the cylinder
%         to round out the shape. In order to calculate interactions between cells, possible overlaps
%         are determined and minimized, thus determining the position values of the next iteration
%         step.
%         BSim does not consider bending forces for individual cells or polar interactions.
%     \item \cite{Gutirrez2017} The gro programming language was designed to simulate the growth of
%         colonies and cell-cell communication. Its “physics computation has been optimized for rigid
%         rod-shaped bodies, like E. coli bacteria” (Gutiérrez et al. 2017). They recognize two types
%         of forces which are acting on the cellular agents: Local forces which are calculated between
%         adjacent bacteria and a global force which pushes bacteria outwards of the colony. The
%         latter of these is a phenomenological implementation of the observed colony expansion and
%         the associated central pressure with it. This assumption may yield incorrect results for
%         sparsely populated cases.
%         The engine is limited to 2D and does not consider polar interactions or bending of the rods.
% \end{itemize}

\paragraph{Applications \& Parameter Estimation}
% \begin{itemize}
%     \item \cite{Winkle2017} Modeling mechanical interations in growing populations of rod-shaped
%         bacteria
%     \item \cite{Doumic2020} "A purely mechanical model with asymmetric features for early
%         morphogenesis of rod-shaped bacteria micro-colony"
%     \begin{itemize}
%         \item Really good paper for referencing
%         \item comparison of distributions; not individual agents
%         \item no bending in model
%         \item uses steric force (see also \cite{Trejo2013})
%     \end{itemize}
%     \item \cite{Grant2014} purposely-built model written in C++
%     \begin{itemize}
%         \item describes bacteria as collection of overlapping spheres
%         \item spheres are coupled by non-linear springs (Euler-Bernoulli dynamic beam theory). This
%             assumption is werid but does not alter their results
%         \item Only model repulsive forces; no attraction, adhesion
%     \end{itemize}
%     \item \cite{Cho2007} only 2D model; no param estimation; based on work done in \cite{Jnsson2005}
%     \item \cite{Storck2014} no bending; 41 parameters for various cases; only 8 parameters taken
%         from literature values or quantified; generate growth rate from normal distribution at each
%         integration step; stochastically equivalient for collection of bacteria but unfounded for
%         individual bactera
%     \item \cite{You2018}
% \end{itemize}

Cho et al.~\cite{Cho2007} extended the work done by~\cite{Jnsson2005} and construct a 2D model to
study self-organization of colonies within restricted geometries of microfluidic devices.
They do not provide any estimates of the parameters that were used.

You et al. assumed a hard-rod model to study the emergence of order within microdomains  of
rod-shaped bacteria~\cite{You2018}.
They were able to show that the hydrodynamic equations of active nematic liquid crystals can serve
as a continuum limit of their model.

Winkle et al. represented cells with "[..] two acially independent cell halves that attach through a
compressible, stiff spring [..]"~\cite{Winkle2017} and leveraged parts of the \texttt{gro}
infrastructure.
However, they only discuss results qualitatively without quantifying any of the used parameters.

Doumic et al. constructed a model which takes into account asymmetries of the individual bacteria
such as length, weight and friction~\cite{Doumic2020}.
They were able to estimate some of the used parameters by comparing distributions of simulated data
to distributions of experimental data.
However, in doing so, no optimization optimization methods were used, but parameters were picked by
hand in an attempt to best fit the observed results.
Furthermore, their model is not able to describe any deformations such as bending of the rods.

Grant et al. designed a custom \texttt{C++} program to simulate individual
bacteria~\cite{Grant2014} which are represented as elastic rods of overlapping spheres.
Their model does not include any attractive forces between agents but is able to describe bending of
the rods in 2D and 3D.
When it comes to the task of estimating parameters of the model, values are picked by hand in order
to mimick the observed experimental results.

Storck et al. constructed a flexible model which is based on particles and springs and thus able to
describe a variety of cells~\cite{Storck2014}.
Since this model incorporates many effects, it consists of $41$ parameters of which only $8$ have
values that are taken from litearture or otherwise quantified.
Another choice of their modeling approach was to stochastically assign growth bursts at each
integration step of the numerical solving process.
When generating from a normal distribution and observing on a collective level, this is
stochastically equivalent to picking and fixing growth rates for each cell from the same
distribution but is not identical when observing individual agents.

\paragraph{The missing Link}
It is clear that researchers are interested in studying mechanical properties of rod-like bacteria
in order to better understand phenomena such as biofilm formation.
The major downside of existing computational models is that there are almost no methods to properly
quantify their parameters which diminishes their explanatory power.
By leveraging a simple assumption about the cellular representation, this work aims to solve this
particular problem.

