\documentclass[10pt]{article}

\usepackage{parskip}
\usepackage[top=2cm,bottom=2cm,left=2.5cm,right=2.5cm]{geometry}
\usepackage{url}

\begin{document}

Jonas Pleyer\textsuperscript{1}, Christian Fleck\textsuperscript{1}\\
\bigskip
\textbf{1} Freiburg Center for Data-Analysis and Modeling, University of Freiburg, Freiburg Germany

February 4, 2026\\
The Editors PLOS Computational Biology\\
Submission of manuscript "Methods for analyzing models of elongated bacteria"

\vspace{1cm}

Dear Editor,

We are pleased to submit our manuscript titled "Methods for analyzing models of elongated bacteria"
for consideration for publication in \textit{PLOS Computational Biology}.

% \textbf{Short Introduction}
% \begin{itemize}
%     \item lack of methods for estimating parameters of ABMs
%     \item in particular with emphasis on individual-based level
%     \item elongated bacteria (such as B.subtilis, E.Coli) are important and have distinct mechanical
%         properties which enables such analyses
%     \item most ABMs do not capture flexible nature of such cells
% \end{itemize}
%
% \textbf{What did we do?}
% \begin{itemize}
%     \item address gap: develop novel methods to estimate parameters
%     \item develop mechanistic, mechanical model of elongated bacteria
%     \item show how single-cell parameters can be estimated, even when involving division events
% \end{itemize}
%
% \textbf{Why is this suited for PLOS?}
% \begin{itemize}
%     \item science: better tools to computationally study bacterial behaviour
%     \item methods: develop even more advanced models while being able to quantify them is new milestone
%     \item novelty: first ever profile-likelihood application to ABMs on single-cell level with division
% \end{itemize}

Individual-Based modeling hase become a popular technique to construct spatial models of elongated
bacteria such as \textit{E.Coli} and \textit{B.Subtilits}.
The distinct mechanical features of these foundational organisms deeply influence cellular
properties such as growth and interactions.
Despite their importance, current computational frameworks not only often lack the ability to
model the flexible nature of these rods, but these tools also fall short when it comes to estimating
the mechanical parameters of these complex models.

In this work, we address this methodological gap by introducing a novel framework that links
mechanistic modeling with quantitative data analysis.
Our contributions are threefold:

\begin{enumerate}
    \item We develop a novel computational model which represents bacteria as deformable rods
        capable of growth, elastic bending and physical interactions
    \item We develop methods to extract positional information of individual cells from segmented
        microscopic images and to compare the numerical output of our simulation with time-series of
        cell masks from experimentally obtained microscopic images.
        These methods allow the parameter estimation of models for cases without and involving
        division events.
    \item We demonstrate that these methods allow us to rigorously estimate the mechanical
        parameters of our model via the profile-likelihood technique on the single-cell level.
        To this end, we apply them in two scenarios thus validating our model against biological
        data and even considering cellular heterogeneity.
\end{enumerate}

We believe that this study is ideally suited for PLOS Computational Biology because it fits
thematically and considerably extends the list of methods with which researchers are able to study
bacterial mechanics in a systematic and quantitative manner even on the single-cell level.

The scientific contributions of this manuscript represent original work and have not been published
or submitted for publication elsewhere.
However we are currently working towards a publication in the Journal of Open Source Software (JOSS)
which publishes software instead of scientific results.
This will include the mentioned software provided in the Github repository
\url{https://github.com/jonaspleyer/cr_mech_coli} together with methods that enable the generation
of realistic microscopic images which are not relevant to this publication.

We have no conflicts of interest to declare.

Thank you for your consideration of our work.
We look forward to hearing from you.

Sincerely,\\
Jonas Pleyer

\end{document}
