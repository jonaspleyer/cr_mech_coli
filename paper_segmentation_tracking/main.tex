\documentclass{article}

\usepackage{arxiv}

\usepackage[utf8]{inputenc} % allow utf-8 input
\usepackage[T1]{fontenc}    % use 8-bit T1 fonts
\usepackage[hidelinks]{hyperref}       % hyperlinks
\usepackage{url}            % simple URL typesetting
\usepackage{booktabs}       % professional-quality tables
\usepackage{amsmath,amssymb,amsthm}
\usepackage{amsfonts}       % blackboard math symbols
\usepackage{float}
\usepackage{nicefrac}       % compact symbols for 1/2, etc.
\usepackage{microtype}      % microtypography
\usepackage{mathrsfs}
\usepackage{onimage}
\usepackage{graphicx}
\usepackage{doi}
\usepackage{acronym}
\usepackage{listings}
\usepackage[mode=buildnew]{standalone}
\usepackage{tikz}
\usepackage{tabularx}
\usepackage{siunitx}
\usepackage{cite}
\usepackage{acronym}
\usepackage{xcolor}

\title{Improving Cell Segmentation and Tracking via Mechanistic Models of
Rod-Shaped Bacteria}

%\date{September 9, 1985}	% Here you can change the date presented in the paper title
%\date{} 					% Or removing it

\author{
    \href{https://orcid.org/0009-0001-0613-7978}{
        % \includegraphics[scale=0.06]{figures/orcid.pdf}
        \hspace{1mm}Student XYZ
    }$^1$
    \href{https://orcid.org/0009-0001-0613-7978}{
        \includegraphics[scale=0.06]{figures/orcid.pdf}
        \hspace{1mm}Jonas Pleyer
    }$^2$
    \href{https://orcid.org/0009-0001-0613-7978}{
        % \includegraphics[scale=0.06]{figures/orcid.pdf}
        \hspace{1mm}Jelena Bratulić
    }$^1$
	% \texttt{jonas.pleyer@fdm.uni-freiburg.de} \\
	%% examples of more authors
	\href{https://orcid.org/0000-0002-6371-4495}{
        \includegraphics[scale=0.06]{figures/orcid.pdf}
        \hspace{1mm}Christian Fleck
    }$^2$
	\href{https://orcid.org/0000-0002-6371-4495}{
        % \includegraphics[scale=0.06]{figures/orcid.pdf}
        \hspace{1mm}Thomas Brox
    }$^1$\\\\
    $^1$University of Freiburg, Germany\\
    $^2$Freiburg Center for Data Analysis, Modeling and AI\\
}

\newacro{abm}[ABM]{Agent-Based Model}
\newacro{ode}[ODE]{Ordinary Differential Equation}
\newacro{pbr}[PBR]{Physically-Based Rendering}

% Uncomment to remove the date
%\date{}

% Uncomment to override  the `A preprint' in the header
\renewcommand{\headeright}{Preprint}
%\renewcommand{\undertitle}{Technical Report}
\renewcommand{\shorttitle}{Improving Cell Segmentation and Tracking via Mechanistic Models of
Rod-Shaped Bacteria}

\usepackage{enumitem}
\setlist{nolistsep}

%%% Add PDF metadata to help others organize their library
%%% Once the PDF is generated, you can check the metadata with
%%% $ pdfinfo template.pdf
\hypersetup{
pdftitle={% TODO
},
pdfsubject={q-bio.NC, q-bio.QM},
pdfauthor={Jonas Pleyer, Christian Fleck},
pdfkeywords={},
}

% Change numbering of equations
% \numberwithin{equation}{section}

% MAKE TITLES IN THEOREMS BOLD
\makeatletter
\def\th@plain{%
  \thm@notefont{}% same as heading font
  \itshape % body font
}
\def\th@definition{%
  \thm@notefont{}% same as heading font
  \normalfont % body font
}
\makeatother

\begin{document}
\maketitle

%###################################################################################################
\begin{abstract}
\end{abstract}

% keywords can be removed
\keywords{Cell Segmentation, Cell Tracking, Mechanistic Modeling, Data Augmentation, Single-Cell}

\pagebreak
\renewcommand{\contentsname}{Table of Contents (remove before submission)}
\tableofcontents
\vfill
\pagebreak

%###################################################################################################
\section{(40\%) Introduction}

The field of biology has become increasingly intertwined with other disciplines such as mathematics,
physics and computation.
With their introduction, many processes have become increasingly automated, thus advancing our
capacity to analyse data at scale.
Many analysis schemes are based on microscopic images of cells which are being observed and possibly
tracked through time and space.
Often, these cells are genetically altered so that effects such as internal production parameters
can be observed from the outside via visual clues.
In order to be able to quantify single-cell effects, it is necessary that we can segment microscopic
images to obtain the location of individual cells.
This challenge of segmenting individual cells is a static problem by nature.
However by considering multiple microscopic images in succession, it is turned into a dynamic
problem of tracking single cells through time and space.
Going even further, we can include division events, thus again increasing the overall complexity of
the problem at hand.

Many algorithms have been conceived that are capable of dealing with these topics and they are being
applied actively within realms such as.
While many earlier iterations targeted 2D situations exclusively, more recent developments are
capable of processing 3D data as well.
Especially this latter subproblem is difficult to tackle since it allows for much larger
complexities within the analyzed dataset.
Furthermore, data in this case is often sparse, thus further complicating efficient training.
But while these tools are essential to get the task done, they can only be as good as the data with
which they are trained.
So far, these datasets are almost exclusively of experimental nature which requires lots of manual
labor in their construction.
The importance of data was also shown in various cases such as \textbf{TODO}.

\begin{itemize}
    \item Compare existing Cell Segmentation+Tracking tools (Cellpose, Omnipose, ..)
\end{itemize}

The goal of this study is to inspect if the process of data generation can be augmented by a
mechanistic single-cell model.
We will study this question along the example of rod-shaped bacteria.
With our recent developments, we have shown how \acp{abm} of rod-shaped bacteria can be fit to
microscopic image time-series, thus allowing us to construct mechanistic models.
From these single-cell models, we will first generate "sterile" snapshots which can then be used to
construct near-realistic brightfield microscopic images and perfect cell masks.
We will mimick known microscopic defects in order to provide realistic images variability that would
also ocurr naturally.
This data will be used to augment existing datasets and asses the effectivenss of this approach when
training existing cell-segmentation and cell-tracking tools on the expanded dataset.
With this procedure we also able to systematically study which of the introduced image defects poses
the biggest challenge for the investigated cell-tracking algorithms.
Furthermore, it will provide experimentalists with information, on which variables are most
important to control within their experimental setup in order to provide the best cell-segmentation
and cell-tracking results.

%###################################################################################################
\section{(70\%) Data Augmentation Strategy}
% --------------------------------------------------------------------------------------------------
\subsection{(90\%) Modeling Approach \& Parameter Estimation}

\begin{itemize}
    \item summarize contents from the other paper which I am preparing
    \item explain on which datasets we fitted the model
\end{itemize}

% --------------------------------------------------------------------------------------------------
\subsection{(80\%) Image Generation}
Using the model of the previous section, we can generate visualizations from its numerical results.
Figure~\ref{fig:progression-image-generation} shows intermediate steps and final results of a
particular generated image.
In the first step (subfigure A), we generate a 3D render which is a collection of meshes and
properties that can be rendered given a set of light sources and a particular view angle.
Using a parallel projection and by assigning uniue colors to individual agents, we can generate cell
masks similar to which would have been generated by a segmentation tool (subfigure B).
Due to the fact that we are utilizing a mechanistic model, we ensure that cells are colored
correctly and that our generated masks are in this sense "perfect".
This is one of the key foundations which allows this investigation in the first place.
Furthermore, since individual agents do not change their labels, the color values of cells are also
carried over correctly between individual time snapshots.
The last image (subfigure C) shows an example of a render where noise effects have been introduced.
These effects are the result of properties of the microscopic appartus, the cellular nature or other
external effects and ocurr naturally within most experimental setups.
We utilize various effects and implement their effect on the resulting image as summarized in
Table~\ref{tabular:microscopic-image-defects}.

Use various effects to make images realistic; mimick real microscopic effects for that
\begin{itemize}
    \item General (white) Noise
    \item Distort Images (lens distortion)
    \item Lighting for rendering in 3D
    \item Shaking for time-series
    \item ...
\end{itemize}

\begin{figure}
    \centering
    % \begin{tikzonimage}[width=0.9\textwidth]{visualization.pdf}
    %     \node at (0.025, 0.975)[anchor=north west, rectangle, draw, minimum width=15pt, minimum height=15pt]{\textbf{A}};
    % \end{tikzonimage}
    \begin{tikzonimage}[width=0.3\textwidth]
        {docs/source/_static/09395645494836445480/raw_pv/000000400.png}
        \node at (0.025, 0.975)[anchor=north west, rectangle, draw, white, minimum width=15pt, minimum height=15pt]{\textbf{A}};
    \end{tikzonimage}
    \begin{tikzonimage}[width=0.3\textwidth]
        {docs/source/_static/09395645494836445480/masks/000000400.png}
        \node at (0.025, 0.975)[anchor=north west, rectangle, draw, white, minimum width=15pt, minimum height=15pt]{\textbf{B}};
    \end{tikzonimage}
    \begin{tikzonimage}[width=0.3\textwidth]
        {docs/source/_static/09395645494836445480/images/000000400.png}
        \node at (0.025, 0.975)[anchor=north west, rectangle, draw, white, minimum width=15pt, minimum height=15pt]{\textbf{C}};
    \end{tikzonimage}
    \caption{
        (A) shows the result from combining sphere and cylinder meshes to obtain the shape of a
        bacterial rod.
        This render contains lighting coming from 2 light sources which thus produces a small glow
        on the upper part of the meshes.
        (B) Unique color values are assigned to each agent and light sources are removed.
        Afterwards, the image is rendered by projecting along the z-axis.
        (C) We use physically-based rendering and assign properties such as roughness and
        reflectivity to generate an initial render.
        Afterwards, we apply white noise to the pixel values of image.
        In the future, we hope to extend this approach in order to construct near-realistic
        microscopic images from synthetic data.
    }
    \label{fig:progression-image-generation}
\end{figure}

\begin{table}[H]
    \begin{tabularx}{\columnwidth}{l l}
        \toprule
        Microscopic effect & Image Transformation\\
        \midrule
        Sensor Noise & White noise on pixel level\\
        Shaking Apparatus & Random spatial translations between time-steps\\
        Lighting & 3D Rendering: Introcue light sources; modify surface of cells; use \ac{pbr}\\
        Cellular Translucency & Make cells partially transparent\\
        Depth of Field & Included in \ac{pbr}\\
        \bottomrule
    \end{tabularx}
    \label{tabular:microscopic-image-defects}
\end{table}

%###################################################################################################
\section{(10\%) Methods}
% --------------------------------------------------------------------------------------------------
\subsection{Data Acquisition}
\begin{itemize}
    \item Use datasets from celltracking challenge
    \item possibly restrict to rod-shaped bacteria for now
\end{itemize}
% --------------------------------------------------------------------------------------------------
\subsection{Testing Methodology}
\begin{enumerate}
    \item Test model on Dataset
    \item Train model on additional data (possibly various configurations)
    \item Test again for same Dataset
\end{enumerate}

%###################################################################################################
\section{(0\%) Results}
% --------------------------------------------------------------------------------------------------
\subsection{Cell Segmentation}
\begin{itemize}
    \item Compare improvement from Cellpose -> Omnipose (generational improvement) with improvement
        in more (augmented) data
    \item Compare how various tools benefit from this approach
    \item How can the variation of generated data improve results?
    \item What is the impact of "noise introduction"? Which effects are more important than others?
\end{itemize}

% --------------------------------------------------------------------------------------------------
\subsection{Cell Tracking}
\begin{itemize}
    \item Use various "graining" of time-series data (larger time-steps between images) to asses the
        importance of this detail
    \item Can the improved ability to track cells through space be generalized to non-rod-shaped
        bacterial tracking?
\end{itemize}

%###################################################################################################
\section{(0\%) Discussion}
\begin{itemize}
    \item How can this be abstracted to other models (such as human stem cells, cocci, etc.)? Refer
        to "blueprint" which was talked about in previous paper
    \item 
\end{itemize}
\section{(0\%) Conclusion}

%###################################################################################################
\section{Data and Code Availability}
All data and code which has been used in this study is available at the github repository
\url{https://github.com/jonaspleyer/cr\_mech\_coli}.
It includes detailed descriptions of how to obtain the used data and bring it into a usable format.
The package which bundles the developed funtionality is published at
\url{https://pypi.org/project/cr-mech-coli/} from which it is readily available as a python
dependency.

\bibliographystyle{IEEEtran}
\bibliography{references}

\appendix
%###################################################################################################
\renewcommand{\thesection}{}
\renewcommand{\thesubsection}{S\arabic{subsection}}
\section{Supplementary Material}

%###################################################################################################
\subsection{Data Preparation}

\end{document}
